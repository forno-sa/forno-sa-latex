\chapter[Concepção]{Concepção}

\section{Introdução}

O tratamento térmico é o conjunto de operações de aquecimento e resfriamento a que são submetidos materiais sob condições controladas, com o objetivo de alterar suas propriedades ou conferir-lhes características mecânicas e estruturais diferentes.

Os tratamentos mais usuais são o recozimento, a normalização, a têmpera, o revenido e o coalescimento. Esses tratamentos são influenciados por alguns fatores que são: o aquecimento, tempo de permanência em determinada temperatura, resfriamento e atmosfera de aquecimento, portanto cada tratamento específico necessita do controle desses fatores.

Portanto um forno de tratamento térmico é de fundamental importância para se realizar esses tratamentos, logo para que o tratamento seja satisfatório, o forno deve ser capaz de variar a temperatura e manter a temperatura de forma satisfatória, de acordo com o tratamento requerido, sem que a temperatura do ambiente externo seja elevado a condições insalubres.

\section{Justificativa}

Uma das atividades de grande importância na ciência dos materiais é o tratamento térmico, que de acordo com Tschiptschin é um ciclo de aquecimento e resfriamento realizado nos metais com o objetivo de alterar as suas propriedades físicas e mecânicas, sem mudar a forma do produto. O tratamento térmico acontece também inadvertidamente, como consequência de um processo de fabricação que cause aquecimento ou resfriamento no metal, como nos casos de soldagem e de forjamento. Esses tratamentos são utilizados em várias situações diferentes como na indústria para alterar características de fabricabilidade, como usinabilidade, estampabilidade ou restauração de ductilidade de metais, o meio acadêmico para o estudo das estruturas microcristalinas dos metais e sua influência nas propriedades do material, e no meio empreendedor como ferramenta para tratamento de vários produtos.

\section{Contextualização}

Um grupo de 15 pessoas de todas as engenharias (Aeroespacial, Automotiva, Eletrônica, Energia e Software) do campus foi montado visando o desenvolvimento de um forno para tratamentos térmicos, desde a teoria até o produto final. Para tal, diversas reuniões presenciais foram realizadas, bem como a utilização de ferramentas de comunicação online e de armazenamento e compartilhamento de arquivos. 

A escolha do tema, bem como das características do forno foram motivadas pela necessidade de ampliar a capacidade da Universidade de Brasília (UnB) de realizar experimentos acerca de materiais. Na idealização do forno, considerou-se a têmpera como a forma de verificação do funcionamento do forno e o aço o material que será tratado. Pretende-se, após a fabricação do forno, disponibilizá-lo para a UnB.

\subsection{Proposta}

Visa-se, ao final do projeto, a fabricação de um forno que possa ser utilizado para o tratamento térmico dos mais diversos materiais, sem que sua superfície externa apresente temperaturas maiores que 60 0C e que seja capaz de elevar sua temperatura interna até 1200 0C de forma controlada. A progressão da temperatura interna poderá ser definida pelo usuário para atender à sua necessidade.

\section{Tecnologias Existentes}

O mercado de fornos para tratamentos térmicos possui uma grande variedade de modelos e fabricantes devido às diferentes aplicações e a importância desses processos na indústria como um todo. Para atender as diferentes aplicações há um grande número de variações nas especificações. Para a realização desse projeto, nos baseamos em alguns modelos citados abaixo com finalidades semelhantes à aplicação do forno a ser projeto nesse trabalho.

\subsection{Forno Mufla para Tratamento Térmico}

\begin{figure}[!h]
	\centering
	\label{forno_mufla}
	\includegraphics[keepaspectratio=true,scale=0.8]{figuras/forno_mufla.JPG}
	\caption{Imagem ilustrativa de forno para tratamento térmico.}
\end{figure}

\subsubsection{Aplicações}

Fornos de câmara para o recozimento, endurecimento, têmpera e envelhecimento (tratamentos térmicos de metalurgia) com circulação de ar.

\subsubsection{Características}

\begin{itemize}
	\item Sensor de Temperatura;
	\item Isolamento Térmico: em fibra cerâmica pré-moldadas;
	\item Estrutura do Forno: totalmente em aço inoxidável com programador de temperatura digital micro processado com programação de rampas e set-point;
	\item Porta localizada na parte frontal com abertura lateral, para o lado esquerdo;
	\item Painel de controle montado com caixa metálica localizado na lateral do forno, separado do corpo do aquecimento;
	\item Chave geral (disjuntor), sinalizador de painel ligado e programador de tempo e temperatura;
	\item Modelos como capacidade entre 20 e 380 litros, potência de 11 a 35kW e temperatura máxima de até 800ºC
	
\end{itemize}

\subsection{Forno para Laboratório}

\begin{figure}[!h]
	\centering
	\label{forno_laboratorio}
	\includegraphics[keepaspectratio=true,scale=0.8]{figuras/forno_laboratorio.JPG}
	\caption{Imagem ilustrativa de Forno para laboratório.}
\end{figure}

\subsubsection{Aplicações}

Forno dedicado à laboratórios de química, cerâmica e metalurgia.

\subsubsection{Características}

\begin{itemize}
	\item Sensor de Temperatura e controlador de temperatura;
	\item Isolamento Térmico: fibra cerâmica pré-moldada e tijolos isolantes superleves;
	\item Controlador de Temperatura;
	\item Estrutura do Forno: Estrutura total em aço inoxidável;
	\item Aquecimento em todas as paredes e porta e sistema de Acionamento de Gás;
	\item Precisão de controle em um ponto de +/- 5ºC; 
	\item Modelos como capacidade entre 10 e 150 litros, potência de 8 a 18kW e temperatura máxima de até 1320ºC
\end{itemize}

\subsection{Forno de Laboratório para Tratamentos Térmicos com gases}

\begin{figure}[!h]
	\centering
	\label{forno_gases}
	\includegraphics[keepaspectratio=true,scale=0.8]{figuras/forno_gases.JPG}
	\caption{Imagem ilustrativa de Forno de laboratório.}
\end{figure}

\subsubsection{Aplicações}

Forno dedicado à tratamentos térmicos de metalurgia com caixa de gases.

\subsubsection{Características}

\begin{itemize}
	\item Caixa de aço inox para tratamento térmico com injeção de gás regulável para tratamentos especiais para todos os tipos de gases;
	\item Isolação térmica de fibra cerâmica pré - moldada e tijolos isolantes super leves;
	\item Estrutura total em aço inoxidável;
	\item Aquecimento em todas as paredes e porta para os fornos de 20 ou mais litros;
	\item Uniforme distribuição de temperatura;
	\item Controlador micro processado, PID, 10 rampas e 10 patamares;
	\item Termopar tipo “S”;
	\item Controle de segurança para excesso de temperatura e quebra de termopar;
	\item Opcional - Comunicação com o microcomputador, mais software gráfico.
	\item Modelos como capacidade entre 40 e 60 litros, potência de 14 a 16kW e temperatura máxima de até 1280ºC
	
\end{itemize}

\section{Requisitos}

\subsection{Requisitos Operacionais}

\begin{itemize}
	\item A temperatura interna do forno será de até 1200 ºC
	\item Temperatura externa máxima de 60ºC
	\item Obter dados do aquecimento
	\item Tratamento térmico para aços
	\item Materiais de pequeno porte
	\item Cadastro de Usuários
	\item Seletor de Temperatura
	\item Iniciar Processo de Tratamento
	\item Histórico de Sessão
	\item Sistema de Segurança
	\item Informações sobre os Processos
	\item Sistema de Login
	\item Variação da temperatura de aquecimento controlada
\end{itemize}

\subsection{Requisitos Técnicos}

\begin{itemize}
	\item Potencia fornecida de 3 KW
	\item NR – 6
	\item NR – 10
	\item NR – 12
	\item NR – 14
	\item NR – 15
	\item Valor das resistencias 
	\item Tijolos Refratários 
	\item Tijolos Refratários 
	\item Dimensionamento interno do forno  
	\item Isolamento com uma manta térmica 
	\item Limite das dimensõe do aço que pode ser tratado no forno 
	\item Sistema de alimentação do forno por eletricidade 
	\item Sensor Termopar tipo K
	\item Raspberry Pi 2
	\item Circuito de condicionamento de sinal
	\item Circuito de controle on/off
	\item Django Framework
	\item REST Framework
	\item Linguagem de Programação Python
	\item ReactJS
\end{itemize}

\section{Objetivo Geral}

Projetar e construir um forno de tratamento térmico para experimentos acadêmicos.

\subsection{Objetivos Específicos}

\begin{itemize}
	\item Temperatura do forno até 1200 ºC com controle liga desliga
	\item Camada externa com temperatura de até 60ºC segundo recomendação da OSHA (Opacional Safety and Health Administration – orgão americano de segurança do trabalhador).
	\item Variação da temperatura de +/- 20ºC
	\item Interface web responsível para controle remoto e relatórios
	\item Sistema de segurança
	
\end{itemize}
